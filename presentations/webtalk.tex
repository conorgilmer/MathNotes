\documentclass[30pt,landscape,footrule]{foils}
\usepackage[english,german]{babel}  % language support for german/english
\usepackage[latin1]{inputenc}       % allow Latin1 characters
\usepackage{ifvtex}
\usepackage{ifpdf}
% If we use vtex, we are likely to create pdf...
\ifvtexpdf\pdftrue\fi
\ifpdf
\usepackage{pause}               % loads also color.sty
\usepackage{background}
\usepackage{graphicx}            % for including graphics
\usepackage{geometry}
\usepackage{hyperref}
\ifvtex\relax
\else
%%\pdfcompresslevel=0
\DeclareGraphicsRule{*}{mps}{*}{}
\fi
\else
\usepackage[dvipdfm]{pause}      % loads also color.sty
\usepackage[dvipdfm]{background}
\usepackage[dvips]{graphicx}
\usepackage[dvips]{geometry}
\usepackage[dvipdfm]{hyperref}
\fi
\usepackage{tabularx}
\usepackage{pp4link}
\usepackage{mpmulti}
\usepackage{amssymb}

\geometry{headsep=3ex,hscale=0.9}
\hypersetup{pdftitle={pp4extensions},
  pdfsubject={Overview of the Internet},
  pdfauthor={Conor Gilmer, IT Marketing Dept., Griffith College Dublin
  <conor.gilmer@gcd.e>},
  pdfkeywords={acrobat, ppower4},
  pdfpagemode={FullScreen},
  colorlinks={true},
  linkcolor={red}
  }


% Set information for title slide and slide footers
\title{A Overview of the Internet}
\author{Conor Gilmer}
\date{October 14, 2009}	


\begin{document}
\maketitle
\parindent 0mm\raggedright
%% Inserting this pauselevel into the Logo we will have the logo appear
%% immediately. But we still need a \pause at the end of the page,
%% otherwise the last page material will also come out first.
%% In highlighted sections we need to keep the link colored for all
%% levels, so better give a wide range for its appearance.
%% Watch your step!
\MyLogo{\pauselevel{highlight =1 :12}Internet and Web / \today \qquad
    \Acrobatmenu{FirstPage}{back to start}\quad}

\foilhead{Contents}
List of Contents
%%{\small
\begin{itemize}
\small{
	\item \toplink{background}{Background} \qquad
	\item \toplink{sengines}{Search Engines} \qquad
	\item \toplink{altengines}{Alternative Search Engines} \qquad
	\item \toplink{software}{Software} \qquad
	\item \toplink{browsers}{Web Browsers} \qquad
	\item \toplink{refernences}{References} \par	
}
\end{itemize}
%%}

\foilhead{Web and the Internet : Timeline}
\toptarget{background}

Chronology\\
Arpanet :- Internetwork communication network that is distributed. nuclear war\\
Protocols\\
Telnet :- remote logging on\\
FTP :- transfer of files\\
Email :-\\
Newsgroups :- foundation of the social element of the internet.\\

Rise in home computing, games, graphics, offices, new tools\\
Concepts :-  Vannevar Bush 1946 ``invented'' the internet\\
Minitel(France Telecom/BT) 1982 - Terminal Online Services
Computserve 1989
Ted Nelson ``hypertext''\\
Tim Berners Lee ``World Wide Web'' 1990ish\\
1994 Netscape, explosion of the web\\
By 1996 increasingly clear that finding things was difficult\\

\foilhead{Software and Tools}
\toptarget{software}
\begin{itemize}
\small{
\item Gopher
\item Usenet News Servers 
\item Telnet :- Windows Telnet, TerraTerm, xTerm or Putty
\item FTP :- Filezilla, WinFTP, CuteFTP
\item Browsers :- lynx, Web Browsers
}
\end{itemize}


\foilhead{Browsers}
\toptarget{browsers}
\definecolor{dimmed}{gray}{0.4}
\pausecolors{magenta}{dimmed}{magenta}
{\color{magenta}
\begin{itemize}
\small{
\item Mosaic\pause
\item Internet Explorer\pause
\item Netscape\pause
\item Mozilla\pause
\item Opera\pause
\item Google Chrome\pause
}
\end{itemize}
}

\foilhead{Search Engines}
\toptarget{sengines}
\definecolor{dimmed}{gray}{0.4}
\pausecolors{magenta}{dimmed}{magenta}

{\color{magenta}
\begin{itemize}
\small{
\item Alta Vista\pause
\item \href{www.ask.com} {Ask Jeeves}\pause
\item Lycos\pause
\item Webcrawler\pause
\item \href{www.yahoo.com} {Yahoo}\pause
\item \href{www.google.com} {Google}\pause
\item MSN, Live and Bing\pause
}
\end{itemize}
}

\foilhead{Alternative Search Engines}
\toptarget{altsengines}
\definecolor{dimmed}{gray}{0.4}
\pausecolors{magenta}{dimmed}{magenta}
{\color{magenta}
\begin{itemize}
\small{
\item Clusty\pause
\item Cuil\pause
\item Digg\pause
\item Wikipedia\pause
}
\end{itemize}
}


\foilhead{The Future}
\toptarget{future}
\definecolor{dimmed}{gray}{0.4}
\pausecolors{magenta}{dimmed}{magenta}
{\color{magenta}
What will Web 3.0 be?
\begin{itemize}
\small{
\item Convergence of Search Engines, Catalogues, Chat Clients, Social Netorking, Twitter, Facebook etc
\item Non Latin domain names! 2010
\item Google SPDY Protocol replaicng http
}
\end{itemize}
}



\foilhead{Thanks for having a look}
\pausecolorreset

The slides should be available to download from 
\href{http://conorg.gcd.ie/webtalk/}{http://conorg.gcd.ie/webtalk}
Please email me with your comments and suggestions about this talk.\\
Thank you for your time and cooperation!

\foilhead{References}
\toptarget{references}
\begin{itemize}
\small {
\item Brink-Budgen, Roy Van Den (2004). ``Critical thinking for students: Learn the skills of critical assessment and effective argument'', Oxford : How to Books.
\item Bush, V. (1945). ``As we may think.'' The Atlantic Monthly.
\item Graham, L. and P. T. Metaxis (2003). ``Of course it's true; I saw it on the Internet!": critical thinking in the Internet era.'' Communications of the ACM 46(5): 70-75.
\item Knight, Charles (2007) `The top 100 search engines Readwriteweb.com'' accessed 16/03/07 http://www.readwriteweb.com/archives/top\_100\_alternative\_search\_engines.php
\item Lucas, I. and N. Helen (2000). ``Defining the Web: The Politics of Search Engines.'' Computer 33(1): 54-62.
\item Metzger, M. J., A. J. Flanagan, et al. (2003). ``College student Web use, perceptions of information credibility, and verification behavior.'' Computers and education 41: 271-290.
\item Morville, Peter (2005) ``Ambient Findability'', O Reilly, Sebastopol.
}
\end{itemize}


\foilhead{Colophon}
This document was created using \LaTeX{}. Initally it was wordprocessed using text editor $<${www.vim.org}$>$ and then rendered into pdf using pdfplatex. \\
Version 2.7 of MiKTeX($<${www.miktex.org}$>$)\\ In researching this topic the Internet browser used was Mozilla Firefox  $<${www.getfirefox.org}$>$ and the Firefox extension Zotero $<${www.zotero.org}$>$ was used  in compiling websites consulted. The Vim plug-in SpellChecker.vim was used to correct my spelling, checking using an American, a British and a Canadian Dictionary.

\copyright 2009 Conor Gilmer  all rights reserved.

\end{document}
