\documentclass{article}

\usepackage{hyperref}
%\hypersetup{colorlinks, %
%	citecolor=black,%
%	filecolor=black,%
%	linkcolor=black,%
%	urlcolor=black,%
%	pdftex}	
\usepackage{graphicx}
\usepackage{tikz}
\usepackage{amssymb,amsmath}
\usepackage{float}
\newcommand{\email}[1]{\texttt{#1}}
\newcommand{\gmat}{GMAT}
\newcommand{\vim}{\href{http://www.vim.org} {www.vim.org} }

\title{\gmat{} Maths Grinds - Basic Calculus}
\author{Conor Gilmer $<$\href{mailto:conor.gilmer@gmail.com}{conor.gilmer@gmail.com}$>$}


\begin{document}

\section{Basic Calculus}
\textit{Calculus} was invented independently by both Leibnetz and Newton
\subsection{Differentiation}
\textbf{Notation}
`
\textbf{Rule - Differentiation of Powers}
\begin{equation}
\begin{split}
y = x^{n} \\
\frac{dy}{dx} = nx^{n-1}
\end{split}
\end{equation}




\textbf{Example 1}

$y = x^{2}\\
\frac{dy}{dx} = 2x^{2-1} = 2x
$


\textbf{Example 2}

$y = x^{2}\\
\frac{dy}{dx} = 2x^{2-1} = 2x
$


\textbf{Example 3}

$y = x\\
\frac{dy}{dx} = 1x^{1-1} = 1.1
$

\textbf{Example 4}

$y = 3\\
\frac{dy}{dx} = 3.0 = 0
$

\textbf{Example 5}

$y = 4x^{3}\\
\frac{dy}{dx} = 4*3*x^{3-1} = 12x^{2}
$


\textbf{Example 6}

$y = 3x + 3x^{4} + 7\\
\frac{dy}{dx} = 3*x^{1-1} + 3*4*x^{4-1} + 7 * 0 \\
\frac{dy}{dx} = 3*x^{0} + 12*x^{3} + 0\\
\frac{dy}{dx} = 3 + 12x^{3} \\
$

\textbf{Example 7}

$y = \frac{1}{x} + 5x - 2x^{\frac{2}{3}} + 3$

Rewrite the indices.

$y = x^{-1} +  5x^{1} - 2x^{\frac{2}{3}} + 3$

Differentiate

$\frac{dy}{dx} = -1*x^{-1-1} +5*x^{1-1} - 2 * \left(\frac{2}{3}\right) * x^{\frac{2}{3} - 1}   + 0 \\
$

$\frac{dy}{dx} = -1*x^{-2} +5*x^{0} -  \left(\frac{4}{3}\right) * x^{\frac{-1}{3}}  \\$

$\frac{dy}{dx} = -x^{-2} + 5 -  \frac{4x^{\frac{-1}{3}}}{3}  \\$




\newpage
\subsection{Integration}
\textit{Integration} the reverse process of \textit{differentiation} think of it as \textit{anti-differentiation}
\textbf{Notation}
$ \int_a^b f(x) $

\textbf{Rules}
 
%\begin{table}
\begin{tabular}{l|l|l}	 	 
Rule & Function & Integral\\
\hline
Multiplication by constant &	 $\int_a^b cf(x) dx$	& c $\int_a^b f(x) dx$ \\
Power Rule ($n \neq -1$)	& $\int_a^b xn dx$ &	$x^{n+1}/(n+1) + C $\\
Sum Rule	 &$\int(f + g) dx$	&$\int_a^b$ f dx + $\int_a^b g dx $\\
Difference Rule &	$\int_a^b (f - g) dx$	& $\int_a^b f dx - \int_a^b g dx $\\
Product Rule & $$ & $ $
%\hline
\end{tabular}
%\caption{\label{Rules}}
%\end{table}

\subsubsection{Integration Power Rule}
\begin{equation}
\int_a^b x dx = \frac{x^{n+1}}{n+1} + C
\end{equation}

\textbf{Example}
Here we will differentiate first before showing the reverse

$ f(x) = y = 3x^{2} + 5x - 3\\
f'(x) = \frac{dy}{dx} = 3(2)x^{2-1} + 5(1)x^{1-1} - 3(0) = 6x + 5 $
So we know what the result of differentiating is so we can see that the reverse process should produce the original expression.

$
\int_a^b f'(x) =   \int_a^b \frac{dy}{dx}\\
\int_a^b 6x + 5 = 3x^{2} + 5x + C \\
$

It Could be any value of $C$.

$\int_a^b 6x + 5 = 3x^{2}+5x-3 \\
\int_a^b 6x + 5 = 3x^{2} + 5x + 11 \\
$

Using the reverse of the rule of powers we can see how we could get  $3x^{2} + 5x $ however we have no way of knowing that $-3$ was there so we would have to use a constant 

\subsubsection{Intergration by Parts}
Involves manipulating the rules into more manageable parts
\begin{enumerate}
\item x is easy to \textit{differentiate}
\item dx is easy to \textit{integrate}
\item sometimes you have to use divide into parts more than once
\end{enumerate} 


\subsubsection{Integration by Substitution}
\textit{Integration by substitution} is reforming the equation in a way which is able to be integrated. 

\end{document}